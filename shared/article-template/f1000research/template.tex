%%%%%%%%%%%%%%%%%%%%%%%%%%%%%%%%%%%%%%%%%%%%%%%%%%%%%%%%%%%%%%%
%
% F1000Research is an Open Research publishing platform for scientists, scholars and clinicians offering rapid publication of articles and other research outputs without editorial bias. All articles benefit from transparent peer review and editorial guidance on making all source data openly available.
%
%%%%%%%%%%%%%%%%%%%%%%%%%%%%%%%%%%%%%%%%%%%%%%%%%%%%%%%%%%%%%%%
%
% This template is for all article types; for information on specific article type requirements please visit https://f1000research.com/for-authors/article-guidelines
%
% For more information on the F1000Research publishing model please see:  https://f1000research.com/about
%
% % % % % % % % % % % % % % % % % % % % % % % 
%
% -- TO DO FOR GTN FORMATTING
%
% This file has been automatically generated by a script.
% Some manual tasks are still needed to be done
%
% - Fill all parts with TO ADD
% - Fill the affiliation and the corresponding author
% - Check the content
% - Use "Fig" to cite "Figure" (see below)
% - Place figure captions after the first paragraph in which they are cited
% - Place tables after the first paragraph in which they are cited
% - Use "Eq" instead of "Equation" for equation citations
% - Add any missing references to the references.bib file and follow the 
%   instructions detailed in \begin{thebibliography} section
%   - 
%
% % % % % % % % % % % % % % % % % % % % % % % % 
%
% In order to ensure smooth and successful processing of your LaTeX manuscript, please follow these guidelines. Before submitting ensure that your PDF compiles correctly, to avoid delays in processing.
% 
% As you prepare your LaTeX manuscript, please bear in mind the following general guidelines.  
% Keep it simple:
% 
% - Keep your LaTeX files as simple as possible; do not use elaborate local macros or highly customized style files. Preferably, use the template provided for formatting your paper.
% - Preferably prepare only one .tex file. 
% - Do not use external style files or packages, except for f1000styles.sty and those packages already referenced in the main.tex template. If you need additional macros, please keep them simple and include them in the .tex document preamble.
% - Source code should be structured so that all .sty and .bst files called by the main .tex file are in the same directory as the main .tex file.
% - AMS math commands are recommended when inserting math equations into your manuscript.
% - When using URLs in the text these should be incorporated as hyperlinks using the \textbackslash hyperref package and \textbackslash href\{\}\{\} function where possible.
% - References to figures and tables within the manuscript should use \textbackslash autoref\{\}
% 
% References:
% -  Reference management systems provide options for exporting bibliographies as BibTEX files (.bib). This template contains an example of such a file, sample.bib, which can be replaced with your own.
% -  Use only the generic \textbackslash cite\{\} command for referencing in the text (like this [1] and this [2]), not other commands built on special macros. Also, make sure that there is no space between reference keynames within the braces (i.e., \textbackslash cite\{refone,reftwo,refthree\}, not \textbackslash cite\{refone, reftwo, refthree\}).
% \end{itemize}
% 
% Submitting your article
% Generate a PDF file of your project and submit this alongside a zip file containing all project files (including the source files, style files, and PDF) using our \href{https://f1000research.com/for-authors/publish-your-research}{online submission form}. 



\documentclass[10pt,a4paper]{article}
\usepackage{f1000research}

%% Default: numerical citations
\usepackage[numbers]{natbib}

%% Uncomment this lines for superscript citations instead
% \usepackage[super]{natbib}

%% Uncomment these lines for author-year citations instead
% \usepackage[round]{natbib}
% \let\cite\citep

\def\tightlist{}

\begin{document}
\pagestyle{fancy}

\title{$title$}
\titlenote{Please provide a concise and specific title that clearly reflects the content of the article.}

% Please list all authors that played a significant role in writing the article. 
% As a guide, authors should refer to the criteria for authorship that have been developed by The International Committee of Medical Journal Editors \href{http://www.icmje.org/recommendations/browse/roles-and-responsibilities/defining-the-role-of-authors-and-contributors.html}{(ICMJE)}. 
$for(author)$\author[1]{$author$}\affil[1]{Address of author-1}$endfor$

\maketitle
\thispagestyle{fancy}

\begin{abstract}
% Abstracts should be up to 300 words and provide a succinct summary of the article. 
% Although the abstract should explain why the article might be interesting, care should be taken not to inappropriately over-emphasise the importance of the work described in the article.
% Citations should not be used in the abstract, and abbreviations, if needed, should be spelled out in full.
$abstract$
\end{abstract}

\section*{\color{f1ROrange}Keywords}
% Please list up to eight relevant keywords that describe the subject of their article. 
% These will improve the visibility of your article.
\clearpage
\pagestyle{fancy}

$body$

\section*{Data and software availability} % Required
% Use this section to provide the raw data that support their findings. Readers should be able to view the raw data, replicate the study, and re-analyse and/or reuse the data (with appropriate attribution). Please take a look at the F1000Research guidelines on \href{https://f1000research.com/for-authors/data-guidelines}{data preparation}.
% Raw data should be uploaded to an approved repository before submission, a list of which can be found on the \href{https://f1000research.com/for-authors/data-guidelines#hosting}{data guidelines page}.

% This section should be completed in the following format:

\subsection*{Source data}

Data used as input for this tutorial can be found at 
\href{$zenodo_link$}{$zenodo_link$}

\subsection*{Training availability}

The tutorial is avalaible at 
\href{$tutorial_link$}{$tutorial_link$}

\section*{Competing interests}
% All financial, personal, or professional competing interests for any of the authors that could be construed to unduly influence the content of the article must be disclosed and will be displayed alongside the article. If there are no relevant competing interests to declare, please add the following: 'No competing interests were disclosed'.

\section*{Grant information}
% Please state who funded the work, whether it is your employer, a grant funder etc. Please do not list funding that you have that is not relevant to this specific piece of research. For each funder, please state the funder’s name, the grant number where applicable, and the individual to whom the grant was assigned.
% If your work was not funded by any grants, please include the section entitled “Grant information” and state: ‘The author(s) declared that no grants were involved in supporting this work’.

\section*{Acknowledgements}
% This section should acknowledge anyone who contributed to the research or the article but who does not qualify as an author based on the criteria provided earlier (e.g. someone or an organization that provided writing assistance). Please state how they contributed; authors should obtain permission to acknowledge from all those mentioned in the Acknowledgements section.
% Please do not list grant funding in this section.

{\small\bibliographystyle{unsrtnat}
\bibliography{references}}

% See this guide for more information on BibTeX:
% http://libguides.mit.edu/content.php?pid=55482&sid=406343

% Please note that this template results in a draft pre-submission PDF document.
% Articles will be professionally typeset when accepted for publication.

% We hope you find the F1000Research LaTex template useful, please contact us if you have any feedback.

\end{document}
